\section{Developing a Model}
\subsection{}

\begin{frame}{Developing a Model}

Model development is not too difficult but can take a long time

\begin{itemize}
\item ARMv7 and RISC-V models developed by students with no prior experience
\item Mostly a task of transcribing instruction pseudocode
\item Can be easy to introduce subtle errors or edge cases
\end{itemize}

\end{frame}

\begin{frame}{Our Workflow}
\end{frame}

\begin{frame}{Continuous Build}

% Jenkins
% Model tests

\end{frame}

\begin{frame}{Instruction Fuzzing}
Fuzzing is a randomized testing method

\begin{itemize}
\item<2-> Test instructions with many inputs against a ground truth
\item<3-> We use QEMU as a ground truth
\item<4-> \alert{Still problems with unspecified/undefined behaviour}
\end{itemize}


\onslide<5->{
We have our own tools for instruction fuzzing (which are linked from GenSim website)
}

\end{frame}

\begin{frame}{Debugging with Tracing}

Once we're confident that our model works, we can run some larger programs

\pause

But they might still go wrong! How do we debug this?

\end{frame}

\begin{frame}{Debugging with Tracing}

ArchSim can produce traces of architectural events
\begin{itemize}
\item Control flow/instructions executed
\item Register reads/writes
\item Memory reads/writes
\end{itemize}

\end{frame}

\begin{frame}{Debugging with Tracing}

\end{frame}

\begin{frame}{Model Optimisation}

% Features
% 

\end{frame}
