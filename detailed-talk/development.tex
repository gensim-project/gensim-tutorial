\section{Developing a Model}
\subsection{}

\begin{frame}{Developing a Model}
So how do we build a model from the ground up?
\end{frame}

\begin{frame}{Developing a Model}

Model development is not too difficult but can take a long time

\begin{itemize}
\item ARMv7 and RISC-V models developed by students with no prior experience
\item Mostly a task of transcribing instruction pseudocode
\item Can be easy to introduce subtle errors or edge cases
\end{itemize}

\end{frame}

\begin{frame}{Our Workflow}

We generally start small and work our way up:
\begin{itemize}
\item Implement enough instructions for a Hello World program
\begin{itemize}
\item (this can be quite a few due to startup/shutdown code)
\end{itemize}
\item Then try small benchmarks (e.g., EEMBC, MyBench)
\item Work up to larger benchmarks (e.g., SPEC)
\end{itemize}

The model should be tested at each step (using fuzzing/small unit tests)

\end{frame}

\begin{frame}{Continuous Build/Test}

We use Jenkins for continuous build. Each commit to GenSim, ArchSim and 
each supported model is built and tested.

% Jenkins
% Model tests

\end{frame}

\begin{frame}{Instruction Fuzzing}
Fuzzing is a randomized testing method

\begin{itemize}
\item<2-> Test instructions with many inputs against a ground truth
\item<3-> We use QEMU as a ground truth
\item<4-> \alert{Still problems with unspecified/undefined behaviour}
\end{itemize}


\onslide<5->{
We have our own tools for instruction fuzzing (which are linked from GenSim website)
}

\end{frame}

\begin{frame}{Debugging with Tracing}

Once we're confident that our model works, we can run some larger programs

\pause

But they might still go wrong! How do we debug this?

\end{frame}


\begin{frame}{Debugging with Tracing}

We can carefully inspect a trace of the simulation and try and see
what went wrong.
\begin{itemize}
\item This can be quite a tedious and error-prone process
\item It is usually necessary to narrow down the issue beforehand
\item Key idea is to spot errors in execution manually
\begin{itemize}
\item Need to be very familiar with the architecture!
\end{itemize}
\end{itemize}

\end{frame}

\begin{frame}{Model Optimisation}

Certain code structures can lead to poor performance:
\begin{itemize}
\item Very complex instructions
\item Non-fixed loop bounds
\item `Hand written' predicates
\item Enablement checks (can be solved using Features)
\end{itemize} 

\end{frame}

\begin{frame}[fragile]{Model Optimisation}
\begin{lstlisting}
if(read_register(FP_ENABLED)) {
	// do something
}
\end{lstlisting}

\begin{lstlisting}
if(__builtin_get_feature(FP_ENABLED)) {
	// do something
}
\end{lstlisting}

\end{frame}
