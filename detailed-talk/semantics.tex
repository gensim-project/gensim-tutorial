\subsection{Semantics Description}

\begin{frame}{Semantics Description}
\usebox{\modelcomponentsbox}
\smash{
	\begin{tikzpicture}[fill opacity=0.3]
		\draw[red,thick] (7.7, 0) rectangle (10.6, 2.3);
		\draw[draw=none] (0,0) rectangle (0, 2.35);
	\end{tikzpicture}
}
\end{frame}


\begin{frame}{GenC}

C-like language used to implement instruction behaviours
\begin{itemize}
\item C-like syntax
\item Only basic types supported
\item Vector types and operations supported natively
\item Reference parameters are supported
\end{itemize}

\end{frame}

\begin{frame}{Architectural Manipulation}

Processor state is manipulated using built-in functions
\begin{itemize}
\item Register accesses
\item Memory access
\item Privilege/ISA modes
\item Floating Point Environment
\end{itemize}

\end{frame}

\begin{frame}[fragile]{Register Access}
Built-in functions available:
\begin{lstlisting}
slot_type read_register(slot_name)
reg_type read_register_bank(bank_name, index)

void write_register(slot_name, slot_type)
void write_register_bank(bank_name, index, reg_type)
\end{lstlisting}

slot\_type depends on the type specified in the System Description

\end{frame}

\begin{frame}[fragile]{Memory Access}

\begin{lstlisting}

void mem_read_[8,16,32,64](address_type, uint[8,16,32,64]&)
void mem_write_[8,16,32,64](address_type, uint[8,16,32,64])

void mem_read_[8,16,32,64]_user(address_type, uint[8,16,32,64]&)
void mem_write_[8,16,32,64]_user(address_type, uint[8,16,32,64])

\end{lstlisting}

`user' variants are used in full-system simulation, where certain 
kernel-mode memory instructions (ldrt/strt in ARM) access memory with
user privileges

\end{frame}

\newsavebox{\vectorboxone}
\begin{lrbox}{\vectorboxone}
\begin{lstlisting}
float[2] fp_vector;
uint32[4] u32_vector;
\end{lstlisting}
\end{lrbox}

\newsavebox{\vectorboxtwo}
\begin{lrbox}{\vectorboxtwo}
\begin{lstlisting}
fp_vector = read_register_bank(FP_SP2, inst.vm);
write_register_bank(U32_V, inst.vm, u32_vector);
\end{lstlisting}
\end{lrbox}

\newsavebox{\vectorboxthree}
\begin{lrbox}{\vectorboxthree}
\begin{lstlisting}
// pairwise addition
result_vector = fp_vector + fp_vector;

// operation by scalar
result_vector = fp_vector + 10.0;
\end{lstlisting}
\end{lrbox}


\begin{frame}[fragile]{Vector Types}

GenC has support for vector types and operations on values of those types
\begin{itemize}
\item \alert<2>{Vectors are declared using array notation}
\item \alert<3>{Vectors can be read and written to register banks}
\item \alert<4>{Arithmetic operations can be performed on vectors}
\end{itemize}

\begin{columns}
\begin{column}{0.99\textwidth}
\centering
\only<2>{
\usebox{\vectorboxone}%
}
\only<3>{
\usebox{\vectorboxtwo}%
}
\only<4>{
\usebox{\vectorboxthree}%
}
\end{column}
\begin{column}{0.01\textwidth}
\vspace{0.3\textheight}
\end{column}
\end{columns}

\end{frame}

\begin{frame}[fragile]{Intrinsics}

There are a large number of intrinsics available for either manipulating
the architectural state or performing other operations. For example:

\begin{itemize}
\item {\ttfamily take\_exception} - trigger an exception
\item {\ttfamily \_\_builtin\_clz} - count leading zeros
\item {\ttfamily bitcast\_float\_u32} - `bitcast' a float to a uint32
\end{itemize}

\end{frame}
