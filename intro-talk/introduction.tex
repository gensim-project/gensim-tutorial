
\section{Introduction}

\begin{frame}{Introduction}
'elevator pitch'
\end{frame}

\begin{frame}{Hands-On Peek}

This tutorial includes a hands-on - participation will require a Linux
machine (several common distributions are supported).

\bigskip

To participate, please go to http://gensim.org/download and install
the dependencies for your particular distribution.

\end{frame}

\begin{frame}{GenSim History}

% Arcsim (?)
% PhD Project
% PAMELA relation
% Open Source

\centering
Ongoing project from University of Edinburgh

\smallskip

\begin{tabular}{r l}
\phantom{0000} - 2011 & Arcsim Simulator \\
2011 - 2014 & GenSim PhD Project \\
2014 - 2018 & PAMELA Project \\
2018 - \phantom{0000}        & Open Source
\end{tabular}

\end{frame}

\begin{frame}{What are we doing with GenSim?}

We use the GenSim toolset in the following research areas:

\begin{itemize}
\item Dynamic Binary Translation
\item GPU Simulation
\item ADL Design and Implementation
\item DSP/VLIW Simulation
\item Cross-architecture Virtualisation
\end{itemize}

\end{frame}

\begin{frame}{Structure of the Tutorial}

\begin{itemize}
\item Talk 1: Introduction (1 Hour)
\begin{itemize}
	\item Brief Introduction to Simulation
	\item Overview of GenSim and Tools
\end{itemize}

\item Hands-On Session (90 Minutes)
\begin{itemize}
	\item Downloading \& Installing GenSim
	\item Using GenSim/ArchSim to perform experiments
\end{itemize}

\item Talk 2: Building a Model (1 Hour)
\begin{itemize}
	\item More detailed look at GenSim
	\item Future plans \& Conclusion
\end{itemize}

\end{itemize}

\end{frame}

\begin{frame}{In This Talk}
	\tableofcontents
\end{frame}	
