% Instructions to download and build gensim
\section{Building GenSim}
\subsection{}

\begin{frame}{Building GenSim}
Building GenSim takes three steps:
\begin{enumerate}
\item Install dependencies
\item Check out GenSim source code
\item Compile
\end{enumerate}
\end{frame}

\begin{frame}{Install dependencies}

GenSim has the following dependencies, which are generally available
and can be installed with your distro's package manager (they may have
different names):

\begin{itemize}
\item autoconf
\item cmake
\item default-jre-headless
\item g++
\item libantlr3c-dev
\item libncurses5-dev
\item make
\item mercurial
\item wget
\item zlib1g-dev
\end{itemize}

\end{frame}

\begin{frame}[fragile]{Check Out Source Code}
GenSim source code is kept in a Mercurial repository on BitBucket. The
code can be obtained by checking out that repository, by running the 
command:

\begin{lstlisting}
hg clone http://bitbucket.org/gensim/gensim
\end{lstlisting}

After the repository is checked out, you can change directory into the repository:

\begin{lstlisting}
cd gensim
\end{lstlisting}

\end{frame}

\begin{frame}[fragile]{Compile}

At this point, everything should be ready for you to compile GenSim!
Simply run

\begin{lstlisting}
make
\end{lstlisting}

... and a short while later GenSim should be compiled. If you have a 
multicore machine and wish to use additional compilation agents, you
can run

\begin{lstlisting}
make -j{N}
\end{lstlisting}

Where \{N\} is the number of build agents to use.

\end{frame}

\begin{frame}{The Built Tools}

You can find the built targets in gensim/build/dist/bin:
\bigskip

\begin{tabular}{ll}
archsim				& The ArchSim simulator \\
archsim-armv7a-user & A script to run ARMv7a binaries \\
gensim              & The GenSim ADL Processing tool \\
TraceCat            & A tool to format binary trace files \\
TraceLess           & A pager for binary trace files \\
TracePCDiff         & A 	diff tool (based on PC) for trace files \\
\end{tabular}

\end{frame}

% Instructions on how to get a prepared VM image (if running windows or mac)

\begin{frame}{Getting a GenSim VM}

If you can't get GenSim to build, or don't have access to one of the
supported Linux distributions, you can try using a pre-prepared 
VirtualBox image.

% TODO: prepare VB image

\end{frame}
