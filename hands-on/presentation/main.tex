% $Header: /Users/joseph/Documents/LaTeX/beamer/solutions/conference-talks/conference-ornate-20min.en.tex,v 90e850259b8b 2007/01/28 20:48:30 tantau $

\documentclass{beamer}

%\documentclass[handout]{beamer}

%\setbeameroption{show notes}

\usepackage{tikz}
\usepackage{xcolor}
\usepackage{tabu}
\usepackage{listings}
\usepackage{pifont}
\usepackage{todonotes}
\usepackage{moresize}

\lstset{basicstyle=\ttfamily\scriptsize}
\lstset{escapeinside={(*}{*)}}

\definecolor{tableShade1}{gray}{0.92}
\definecolor{tableShade2}{gray}{0.97} 

% This file is a solution template for:

% - Talk at a conference/colloquium.
% - Talk length is about 20min.
% - Style is ornate.



% Copyright 2004 by Till Tantau <tantau@users.sourceforge.net>.
%
% In principle, this file can be redistributed and/or modified under
% the terms of the GNU Public License, version 2.
%
% However, this file is supposed to be a template to be modified
% for your own needs. For this reason, if you use this file as a
% template and not specifically distribute it as part of a another
% package/program, I grant the extra permission to freely copy and
% modify this file as you see fit and even to delete this copyright
% notice. 


\mode<presentation>
{
	\usepackage[icsa,logoseparator,nonav,slidecount,secheadingsfooter]{beamerthemeinformatics}
	% or ...
	
%	\setbeamercovered{transparent}
	% or whatever (possibly just delete it)
}

\setbeamercolor{alerted text}{fg=red}
\setbeamerfont{alerted text}{series=\bfseries}

\newcommand{\HWPlaceholder}[3]{
	\begin{figure}
	\begin{tikzpicture}
		\node[minimum height=#3, minimum width=#2,draw] at (0,0) {#1};
	\end{tikzpicture}
	\end{figure}
}

\usepackage[english]{babel}
% or whatever

\usepackage[latin1]{inputenc}
% or whatever

\usepackage{times}
\usepackage[T1]{fontenc}
% Or whatever. Note that the encoding and the font should match. If T1
% does not look nice, try deleting the line with the fontenc.

\title % (optional, use only with long paper titles)
{GenSim}

\subtitle{Hands-On Session}

\author % (optional, use only with lots of authors)
{Harry Wagstaff, Tom Spink, Bruno Bodin, Bjoern Franke}
% - Give the names in the same order as the appear in the paper.
% - Use the \inst{?} command only if the authors have different
%   affiliation.

\institute % (optional, but mostly needed)
{
	Institute for Computing Systems Architecture \\
	University of Edinburgh
}
% - Use the \inst command only if there are several affiliations.
% - Keep it simple, no one is interested in your street address.

\date % (optional, should be abbreviation of conference name)
{April 2018}
% - Either use conference name or its abbreviation.
% - Not really informative to the audience, more for people (including
%   yourself) who are reading the slides online

%\subject{Theoretical Computer Science}
% This is only inserted into the PDF information catalog. Can be left
% out. 



% If you have a file called "university-logo-filename.xxx", where xxx
% is a graphic format that can be processed by latex or pdflatex,
% resp., then you can add a logo as follows:

% \pgfdeclareimage[height=0.5cm]{university-logo}{university-logo-filename}
% \logo{\pgfuseimage{university-logo}}



% Delete this, if you do not want the table of contents to pop up at
% the beginning of each subsection:
%\AtBeginSubsection[]
%{
%	\begin{frame}<beamer>{Outline}
%		\tableofcontents[currentsection,currentsubsection]
%	\end{frame}
%}

% If you wish to uncover everything in a step-wise fashion, uncomment
% the following command: 

%\beamerdefaultoverlayspecification{<+->}

\begin{document}
	
\begin{frame}
  \titlepage
\end{frame}

% Structuring a talk is a difficult task and the following structure
% may not be suitable. Here are some rules that apply for this
% solution:
	
% - Exactly two or three sections (other than the summary).
% - At *most* three subsections per section.
% - Talk about 30s to 2min per frame. So there should be between about
% 15 and 30 frames, all told.
	
% - A conference audience is likely to know very little of what you
% are going to talk about. So *simplify*!
% - In a 20min talk, getting the main ideas across is hard
% enough. Leave out details, even if it means being less precise than
% you think necessary.
% - If you omit details that are vital to the proof/implementation,
% just say so once. Everybody will be happy with that.

% what are we going to be doing in this session
\section{Introduction}
\subsection{}

\begin{frame}{Introduction}

In this session, you will learn:
\begin{itemize}
\item How to build GenSim from source
\item How to perform simulations using ArchSim
\item How to collect useful information from ArchSim
\item How to add to existing GenSim models
\end{itemize}

\end{frame}

% what will you need to participate in the hands-on session

\begin{frame}{Introduction}

GenSim tested on the following Linux distributions:
\begin{itemize}
\item Ubuntu 16.04
\item Ubuntu 18.04
\item Fedora 26
\item Fedora 27
\item ArchLinux
\item Debian 9.3
\end{itemize}

If you have another distribution, or another OS, you may need to use
a GenSim VM image to take part in this session.

\end{frame}


% Instructions to download and build gensim
\section{Building GenSim}
\subsection{}

\begin{frame}{Building GenSim}
Building GenSim takes three steps:
\begin{enumerate}
\item Install dependencies
\item Check out GenSim source code
\item Compile
\end{enumerate}
\end{frame}

\begin{frame}{Install dependencies}

GenSim has the following dependencies, which are generally available
and can be installed with your distro's package manager (they may have
different names):

\begin{itemize}
\item autoconf
\item cmake
\item default-jre-headless
\item g++
\item libantlr3c-dev
\item libncurses5-dev
\item make
\item mercurial
\item wget
\item zlib1g-dev
\end{itemize}

\end{frame}

\begin{frame}[fragile]{Check Out Source Code}
GenSim source code is kept in a Mercurial repository on BitBucket. The
code can be obtained by checking out that repository, by running the 
command:

\begin{lstlisting}
hg clone http://bitbucket.org/gensim/gensim
\end{lstlisting}

After the repository is checked out, you can change directory into the repository:

\begin{lstlisting}
cd gensim
\end{lstlisting}

\end{frame}

\begin{frame}[fragile]{Compile}

At this point, everything should be ready for you to compile GenSim!
Simply run

\begin{lstlisting}
make
\end{lstlisting}

... and a short while later GenSim should be compiled. If you have a 
multicore machine and wish to use additional compilation agents, you
can run

\begin{lstlisting}
make -j{N}
\end{lstlisting}

Where \{N\} is the number of build agents to use.

\end{frame}

\begin{frame}{The Built Tools}

You can find the built targets in gensim/build/dist/bin:
\bigskip

\begin{tabular}{ll}
archsim				& The ArchSim simulator \\
archsim-armv7a-user & A script to run ARMv7a binaries \\
gensim              & The GenSim ADL Processing tool \\
TraceCat            & A tool to format binary trace files \\
TraceLess           & A pager for binary trace files \\
TracePCDiff         & A 	diff tool (based on PC) for trace files \\
\end{tabular}

\end{frame}

% Instructions on how to get a prepared VM image (if running windows or mac)

\begin{frame}{Getting a GenSim VM}

If you can't get GenSim to build, or don't have access to one of the
supported Linux distributions, you can try using a pre-prepared 
VirtualBox image.

% TODO: prepare VB image

\end{frame}


% Instructions on how to run some prepared ARM binaries
\section{Running ArchSim}
\subsection{}

\newsavebox{\runningarchsimbox}
\begin{lrbox}{\runningarchsimbox}
\begin{lstlisting}[basicstyle=\tiny\ttfamily]
 		Archsim: The Edinburgh High Speed (EHS) Simulator
 			University of Edinburgh (c) 2017
 				Revision : 5979651a74c9+ tip
 				Configuration: "Debug (Opt)"

Hello, world!

\end{lstlisting}
\end{lrbox}

\begin{frame}[fragile]{Running ArchSim}
Now that everything is built, it's time to run some simulations!

\begin{lstlisting}
gensim/build/dist/bin/archsim-armv7a-user $(mat)/running/hello
\end{lstlisting}

\only<2>{
Output:

\usebox{\runningarchsimbox}
}

\end{frame}

% Instructions on how to run custom simulations
\begin{frame}{Advanced Configuration}

Although the default configuration is fine for straightforward simulation,
if you want to simulate other architectures or systems then you need to
call ArchSim directly.

\end{frame}

\begin{frame}[fragile]{Advanced Configuration}

{\ttfamily archsim-armv7a-user} is a script which calls ArchSim. Let's
have a closer look at it, and in particular the line which actually
calls ArchSim:

\centering
\begin{minipage}{\textwidth}
\centering
\begin{lstlisting}
$ARCHSIM -m arm-user -s armv7a --mode interp -l contiguous -e $ELF
\end{lstlisting}

\only<2>{
\smash{
	\begin{tikzpicture}[fill opacity=0.3]
		\draw[red,thick] (1.3, 0.5) rectangle (3.3, 1);
		\draw[draw=none] (0,0) rectangle (0, 1);
	\end{tikzpicture}%
}
}
\only<3>{
\smash{
	\begin{tikzpicture}[fill opacity=0.3]
		\draw[red,thick] (3.4, 0.5) rectangle (5.0, 1);
		\draw[draw=none] (0,0) rectangle (0, 1);
	\end{tikzpicture}%
}
}
\only<4>{
\smash{
	\begin{tikzpicture}[fill opacity=0.3]
		\draw[red,thick] (5.1, 0.5) rectangle (7.4, 1);
		\draw[draw=none] (0,0) rectangle (0, 1);
	\end{tikzpicture}%
}
}
\only<5>{
\smash{
	\begin{tikzpicture}[fill opacity=0.3]
		\draw[red,thick] (7.45, 0.5) rectangle (9.8, 1);
		\draw[draw=none] (0,0) rectangle (0, 1);
	\end{tikzpicture}%
}
}
\only<6>{
\smash{
	\begin{tikzpicture}[fill opacity=0.3]
		\draw[red,thick] (9.8, 0.5) rectangle (11.1, 1);
		\draw[draw=none] (0,0) rectangle (0, 1);
	\end{tikzpicture}%
}
}

\end{minipage}

\begin{minipage}{\textwidth}
\only<2>{The Emulation Model - How is the binary loaded? What happens when an exception occurs?}
\only<3>{The Guest Architecture - What guest architecture should be used?}
\only<4>{The Execution Mode - at the moment should always be set to `interp'}
\only<5>{The Memory Model - How should underlying memory be stored?}
\only<6>{The target binary - which binary should be loaded for simulation?}
\end{minipage}

\end{frame}

\begin{frame}[fragile]{Advanced Configuration}

Let's try running with a custom configuration. We'll add the {\ttfamily ---verbose} flag, which will cause simulation information to be printed:

\begin{lstlisting}[basicstyle=\ttfamily\ssmall]
archsim -m arm-user -s armv7a --mode interp -l contiguous --verbose -e $(mat)/running/hello
\end{lstlisting}

This will cause some simulation statistics to be printed, including
speed, instruction count, simulation time, etc.

\end{frame}

% Instructions on how to run prepared RISC-V binaries
\begin{frame}{RISC-V}

\end{frame}


% Instructions on how to get verbose/profiling information
\section{Profiling}
\subsection{}

\begin{frame}{Obtaining Simulation Statistics}

ArchSim provides features to collect simulation statistics in a variety
of ways. We've already seen how basic statistics can be collected, but
they're mainly about the simulator rather than the simulated binary.

Additional profiling features include histograms of:
\begin{itemize}
\item Instruction types executed
\item PC frequencies
\item Instruction code frequencies
\end{itemize}

\end{frame}

\begin{frame}{Profiling Information}

These histograms can be obtained with the following command-line flags:

\begin{itemize}
\item Instruction types: {\ttfamily ---verbose ---profile}
\item PC frequencies: {\ttfamily ---verbose ---profile-pc}
\item Instruction codes: {\ttfamily ---verbose ---profile-ir}
\end{itemize}

The IR and PC histograms are written out to ir\_freq.out and pc\_freq.out.

\end{frame}

\begin{frame}[fragile]{Tracing}

If these features do not provide enough detail, then a full trace can
be obtained and processed.

The following flags can be added to enable tracing:

\begin{lstlisting}
--trace --trace-file $(output-file)
\end{lstlisting}

This will write a trace out to a file called \$(output-file).

\end{frame}

\begin{frame}[fragile]{Tracing}

The TraceCat and TraceLess tools can be used to view the traces. 
Alternatively, a custom tool can be used to perform other analysis or
trace-based simulation.

\begin{lstlisting}
build/dist/bin/TraceLess $(trace file)
\end{lstlisting}

This tool behaves like {\ttfamily less} (although not quite as feature rich)

\end{frame}


% Instructions on how to add dot product instruction to RISC-V model
\section{Modifying an Existing Model}

\begin{frame}{Modifying an Existing Model}
Now, let's try modifying the RISC-V model.
\end{frame}

\begin{frame}[fragile]{The New Instruction}

Let's an an integer dot product instruction to the instruction set. 
The instruction will operate on two pairs of registers:

\begin{lstlisting}
$(rd) = $(rs1) * $(rs1+1) + $(rs2) * $(rs2 + 1)
\end{lstlisting}

RISC-V has instruction space reserved for customisation, which this 
instruction will use.

\end{frame}

\begin{frame}{Encoding the Instruction}

% models/risc-v/riscv_isa.ac

% use format Type_R

% opcode == 0b0001011

% funct3 = 0

% funct7 = 0

First, we need to settle on the encoding for the instruction. The 32-bit
RISC-V opcode space is divided into 28 subspaces, several of
which are reserved for custom extensions. 

\bigskip

We will use the custom-0 space, which uses the 0b0001011 opcode.

\smallskip

We'll use the R-Type format, since we're doing an operation on registers.

\smallskip

To keep things simple, we'll fill 0s in to the instruction function
fields.

\end{frame}

\begin{frame}{Instruction Syntax}

Let's start editing the model! Open up {\ttfamily gensim/models/risc-v/riscv\_isa.ac}.

This is the Syntax Description file for the base RISC-V instruction set.
To keep things simple, we'll edit this file rather than creating a new 
file for our extension. 

\end{frame}

\begin{frame}[fragile]{Instruction Syntax - Adding the new Instruction}

If you scroll to line 39, you'll find a section of the file which 
declares which instructions are available in the instruction set. Each
instruction is attached for a format, and they are declared in a 
`template'-like manner.

\smallskip

Line 52 specifies the instructions for the R-Type format. Add a 
{\ttfamily dotproduct} entry to the end of this line, like this:

\begin{lstlisting}
ac_instr<Type_R> add,sub,sll,slt,sltu,xor,srl,sra,or,and,dotproduct;
\end{lstlisting}

\end{frame}

\begin{frame}[fragile]{Instruction Syntax - Declaring the Instruction Behaviour}
From Line 74, the instruction semantic behaviours are declared. We need
to add a new line for our dotproduct instruction

\begin{lstlisting}
ac_behaviour dotproduct;
\end{lstlisting}

\end{frame}

\begin{frame}[fragile]{Instruction Syntax - Decoding the new Instruction}

From Line 124, the decoding for each instruction is described. For our
new dotproduct instruction, we'll need to add a couple of new lines. 
Add the following lines to the section (the position doesn't matter):

\begin{lstlisting}
dotproduct.set_decoder(opcode=0x0b, funct3=0x0, funct7=0x0);
dotproduct.set_behaviour(dotproduct);
\end{lstlisting}

\end{frame}

\begin{frame}{The Instruction Semantics}

That's all we need to decode the instruction. Now, we need to specify
the semantics of the instruction. Open {\ttfamily gensim/models/risc-v/execute.riscv}.

\bigskip

In this file, the semantic behaviour of each RISC-V instruction is specified.

\end{frame}

\begin{frame}[fragile]{The Instruction Semantics}

Add the following text to the start of the file:

\begin{lstlisting}
execute(dotproduct) {
	uint32 rs1a = read_register_bank(GPR, inst.rs1);
	uint32 rs1b = read_register_bank(GPR, inst.rs1+1);
	uint32 rs2a = read_register_bank(GPR, inst.rs2);
	uint32 rs2b = read_register_bank(GPR, inst.rs2+1);
	
	uint32 result = (rs1a * rs1b) + (rs2a * rs2b);
	
	write_register_bank(GPR, inst.rd, result);
}
\end{lstlisting}

\end{frame}

\begin{frame}{Building the model}
We've now added enough to the model to decode and execute our new 
instruction, so we need to rebuild the model. Go to the root of the
gensim repository and run {\ttfamily make}. If the system builds with no errors,
then you can now run simulations using the new instruction!
\end{frame}

\begin{frame}[fragile]{Using The Instruction}

Now that we have built the model containing the new instruction, we can
run a program which uses it! The {\ttfamily \$(mat)/riscv-dotproduct} directory contains
an example program which uses this instruction. We can run this program to
test our new instruction:

\begin{lstlisting}[basicstyle=\ttfamily\ssmall]
archsim -s riscv -m riscv-user -l contiguous -e $(mat)/riscv-dotproduct/dotproduct
\end{lstlisting}

\bigskip

\begin{lstlisting}
Operands: (1804289383, 846930886) (1681692777, 1714636915)
C Result: 4088578517
Asm Result: 4088578517
\end{lstlisting}

\end{frame}


\section{Conclusion}
\subsection{}

\begin{frame}{Conclusion}
	\centering
	Thanks for coming!
	
	\bigskip
	
	harry.wagstaff@gmail.com
	
	general@gensim.org
\end{frame}


\end{document}
