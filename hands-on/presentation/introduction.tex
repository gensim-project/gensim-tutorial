% what are we going to be doing in this session
\section{Introduction}
\subsection{}

\begin{frame}{Introduction}

In this session, you will learn:
\begin{itemize}
\item How to build GenSim from source
\item How to perform simulations using ArchSim
\item How to collect useful information from ArchSim
\item How to add to existing GenSim models
\end{itemize}

\end{frame}

% what will you need to participate in the hands-on session

\begin{frame}{Introduction}

GenSim tested on the following Linux distributions:
\begin{itemize}
\item Ubuntu 16.04
\item Ubuntu 18.04
\item Fedora 26
\item Fedora 27
\item ArchLinux
\item Debian 9.3
\end{itemize}

If you have another distribution, or another OS, you may need to use
a GenSim VM image to take part in this session.

\end{frame}

\begin{frame}[fragile]{Materials}

To complete this session, you will also need to simulate some binaries.
These binaries are pre-built, and can be obtained from the GenSim Tutorial
downloads section at:

\url{http://bitbucket.org/gensim/gensim-tutorial/downloads}

Extract this archive to a directory of your choice. In this slide deck,
we'll be referring to this directory as {\ttfamily \$(mat)}, e.g., 
\begin{lstlisting}
$(mat)/running-archsim/hello
\end{lstlisting}

\end{frame}
