% Instructions on how to run some prepared ARM binaries
\section{Running ArchSim}
\subsection{}

\newsavebox{\runningarchsimbox}
\begin{lrbox}{\runningarchsimbox}
\begin{lstlisting}[basicstyle=\tiny\ttfamily]
 		Archsim: The Edinburgh High Speed (EHS) Simulator
 			University of Edinburgh (c) 2017
 				Revision : 5979651a74c9+ tip
 				Configuration: "Debug (Opt)"

Hello, world!

\end{lstlisting}
\end{lrbox}

\begin{frame}[fragile]{Running ArchSim}
Now that everything is built, it's time to run some simulations!

\begin{lstlisting}
gensim/build/dist/bin/archsim-armv7a-user $(mat)/running/hello
\end{lstlisting}

\only<2>{
Output:

\usebox{\runningarchsimbox}
}

\end{frame}

% Instructions on how to run custom simulations
\begin{frame}{Advanced Configuration}

Although the default configuration is fine for straightforward simulation,
if you want to simulate other architectures or systems then you need to
call ArchSim directly.

\end{frame}

\begin{frame}[fragile]{Advanced Configuration}

{\ttfamily archsim-armv7a-user} is a script which calls ArchSim. Let's
have a closer look at it, and in particular the line which actually
calls ArchSim:

\centering
\begin{minipage}{\textwidth}
\centering
\begin{lstlisting}
$ARCHSIM -m arm-user -s armv7a --mode interp -l contiguous -e $ELF
\end{lstlisting}

\only<2>{
\smash{
	\begin{tikzpicture}[fill opacity=0.3]
		\draw[red,thick] (1.3, 0.5) rectangle (3.3, 1);
		\draw[draw=none] (0,0) rectangle (0, 1);
	\end{tikzpicture}%
}
}
\only<3>{
\smash{
	\begin{tikzpicture}[fill opacity=0.3]
		\draw[red,thick] (3.4, 0.5) rectangle (5.0, 1);
		\draw[draw=none] (0,0) rectangle (0, 1);
	\end{tikzpicture}%
}
}
\only<4>{
\smash{
	\begin{tikzpicture}[fill opacity=0.3]
		\draw[red,thick] (5.1, 0.5) rectangle (7.4, 1);
		\draw[draw=none] (0,0) rectangle (0, 1);
	\end{tikzpicture}%
}
}
\only<5>{
\smash{
	\begin{tikzpicture}[fill opacity=0.3]
		\draw[red,thick] (7.45, 0.5) rectangle (9.8, 1);
		\draw[draw=none] (0,0) rectangle (0, 1);
	\end{tikzpicture}%
}
}
\only<6>{
\smash{
	\begin{tikzpicture}[fill opacity=0.3]
		\draw[red,thick] (9.8, 0.5) rectangle (11.1, 1);
		\draw[draw=none] (0,0) rectangle (0, 1);
	\end{tikzpicture}%
}
}

\end{minipage}

\begin{minipage}{\textwidth}
\only<2>{The Emulation Model - How is the binary loaded? What happens when an exception occurs?}
\only<3>{The Guest Architecture - What guest architecture should be used?}
\only<4>{The Execution Mode - at the moment should always be set to `interp'}
\only<5>{The Memory Model - How should underlying memory be stored?}
\only<6>{The target binary - which binary should be loaded for simulation?}
\end{minipage}

\end{frame}

\begin{frame}[fragile]{Advanced Configuration}

Let's try running with a custom configuration. We'll add the {\ttfamily ---verbose} flag, which will cause simulation information to be printed:

\begin{lstlisting}[basicstyle=\ttfamily\ssmall]
archsim -m arm-user -s armv7a --mode interp -l contiguous --verbose -e $(mat)/running/hello
\end{lstlisting}

This will cause some simulation statistics to be printed, including
speed, instruction count, simulation time, etc.

\end{frame}

% Instructions on how to run prepared RISC-V binaries
\begin{frame}{RISC-V}

\end{frame}
