% Instructions on how to add dot product instruction to RISC-V model
\section{Modifying an Existing Model}

\begin{frame}{Modifying an Existing Model}
Now, let's try modifying the RISC-V model.
\end{frame}

\begin{frame}[fragile]{The New Instruction}

Let's an an integer dot product instruction to the instruction set. 
The instruction will operate on two pairs of registers:

\begin{lstlisting}
$(rd) = $(rs1) * $(rs1+1) + $(rs2) * $(rs2 + 1)
\end{lstlisting}

RISC-V has instruction space reserved for customisation, which this 
instruction will use.

\end{frame}

\begin{frame}{Encoding the Instruction}

% models/risc-v/riscv_isa.ac

% use format Type_R

% opcode == 0b0001011

% funct3 = 0

% funct7 = 0

First, we need to settle on the encoding for the instruction. The 32-bit
RISC-V opcode space is divided into 28 subspaces, several of
which are reserved for custom extensions. 

\bigskip

We will use the custom-0 space, which uses the 0b0001011 opcode.

\smallskip

We'll use the R-Type format, since we're doing an operation on registers.

\smallskip

To keep things simple, we'll fill 0s in to the instruction function
fields.

\end{frame}

\begin{frame}{Instruction Syntax}

Let's start editing the model! Open up {\ttfamily gensim/models/risc-v/riscv\_isa.ac}.

This is the Syntax Description file for the base RISC-V instruction set.
To keep things simple, we'll edit this file rather than creating a new 
file for our extension. 

\end{frame}

\begin{frame}[fragile]{Instruction Syntax - Adding the new Instruction}

If you scroll to line 39, you'll find a section of the file which 
declares which instructions are available in the instruction set. Each
instruction is attached for a format, and they are declared in a 
`template'-like manner.

\smallskip

Line 52 specifies the instructions for the R-Type format. Add a 
{\ttfamily dotproduct} entry to the end of this line, like this:

\begin{lstlisting}
ac_instr<Type_R> add,sub,sll,slt,sltu,xor,srl,sra,or,and,dotproduct;
\end{lstlisting}

\end{frame}

\begin{frame}[fragile]{Instruction Syntax - Declaring the Instruction Behaviour}
From Line 74, the instruction semantic behaviours are declared. We need
to add a new line for our dotproduct instruction

\begin{lstlisting}
ac_behaviour dotproduct;
\end{lstlisting}

\end{frame}

\begin{frame}[fragile]{Instruction Syntax - Decoding the new Instruction}

From Line 124, the decoding for each instruction is described. For our
new dotproduct instruction, we'll need to add a couple of new lines. 
Add the following lines to the section (the position doesn't matter):

\begin{lstlisting}
dotproduct.set_decoder(opcode=0x0b, funct3=0x0, funct7=0x0);
dotproduct.set_behaviour(dotproduct);
\end{lstlisting}

\end{frame}

\begin{frame}{The Instruction Semantics}

That's all we need to decode the instruction. Now, we need to specify
the semantics of the instruction. Open {\ttfamily gensim/models/risc-v/execute.riscv}.

\bigskip

In this file, the semantic behaviour of each RISC-V instruction is specified.

\end{frame}

\begin{frame}[fragile]{The Instruction Semantics}

Add the following text to the start of the file:

\begin{lstlisting}
execute(dotproduct) {
	uint32 rs1a = read_register_bank(GPR, inst.rs1);
	uint32 rs1b = read_register_bank(GPR, inst.rs1+1);
	uint32 rs2a = read_register_bank(GPR, inst.rs2);
	uint32 rs2b = read_register_bank(GPR, inst.rs2+1);
	
	uint32 result = (rs1a * rs1b) + (rs2a * rs2b);
	
	write_register_bank(GPR, inst.rd, result);
}
\end{lstlisting}

\end{frame}

\begin{frame}{Building the model}
We've now added enough to the model to decode and execute our new 
instruction, so we need to rebuild the model. Go to the root of the
gensim repository and run {\ttfamily make}. If the system builds with no errors,
then you can now run simulations using the new instruction!
\end{frame}

\begin{frame}[fragile]{Using The Instruction}

Now that we have built the model containing the new instruction, we can
run a program which uses it! The {\ttfamily \$(mat)/riscv-dotproduct} directory contains
an example program which uses this instruction. We can run this program to
test our new instruction:

\begin{lstlisting}[basicstyle=\ttfamily\ssmall]
archsim -s riscv -m riscv-user -l contiguous -e $(mat)/riscv-dotproduct/dotproduct
\end{lstlisting}

\bigskip

\begin{lstlisting}
Operands: (1804289383, 846930886) (1681692777, 1714636915)
C Result: 4088578517
Asm Result: 4088578517
\end{lstlisting}

\end{frame}
